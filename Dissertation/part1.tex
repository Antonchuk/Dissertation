\chapter{Методика атомно-зондовая томография}\label{ch:ch1}

В основе атомно-зондовой томографии(АЗТ) лежит три основных принципа. Один принцип - возможность испарения материала с помощью постоянного электрического поля (в некоторых случаях с импульсной составляющей). Второй - время-пролетная масс-спектрометрия. Третий - проекционный принцип восстановления трехмерного взаиморасположения атомов. Какие-ото общие слова про томограф (составляющие, потатомное испарение)

\section{Полевое испарение в атомно-зондовой томографии}\label{sec:ch1/sec1}

\nomenclature{\(АЗТ\)}{атомно-зондовая томография}
В атомно-зондовой томографии используется принцип полевого испарения для того, чтобы последовательно удалять атомы от поверхности образца. Испарение происходит путем ионизации атомов с поверхности образца с последующим ускорением в электрическом поле. Ионизация происходит из-за комбинации воздействий постоянного напряжения и высоковольтного или лазерного импульса, приложенного к поверхности образца. Первой методикой, использовавшей данные механизмы, была полевая ионная микроскопия [ссылка]. В последствии атомно-зондовая томография заимствовала часть методов и з автоионной микроскопии.
В настоящий момент механизм испарения отдельных атомов с помощью электрического поля не описан достаточно полно и точно. Обычно, для объяснения данного механизма используются простые термодинамические соображения. Одно из первых описаний испарения с помощью электрического поля представил Миллер \cite{Muller56}. Контролируемое испарение атома представлялось как прохождение атомом энергетического барьера на поверхности образца. В модели Мюллера считается, что испаренный атом имеет такую же энергию, как атом, абсорбированный на поверхности образца. В случае отсутствия приложенного к образцу поля, энергия $Q_0$, необходимая для удаления атома от поверхности и его n-кратной ионизации, может быть оценена с помощью цикла Борна-Габера.

\begin{equation}
	\label{eq:equation1}
	Q_0 = \Lambda + \sum_{1}^{n} I_i -n\phi_e
\end{equation}

где $\Lambda$ - энергия сублимации, $I_i$ - энергия ионизации, $\phi_e$ - работа выхода электрона, n - количество электронов. Значения энергий ионизации, сублимации и работа выхода электрона известны дя атомов большинства сортов и могут считаться табличными. Ниже представлены графики для энергетического барьера атома на поверхности образца.

\begin{figure}[ht]
	\begin{minipage}[b][][b]{0.49\textwidth}\centering
		\includegraphics[width=\textwidth]{mullerenergy} \\ а)
	\end{minipage}
	%\hfill
	\begin{minipage}[b][][b]{0.49\textwidth}\centering
		\includegraphics[width=\textwidth]{mullerenergy} \\ б)
	\end{minipage}
	\caption{Диаграмма потенциальной энергии атома без приложенного электрического поля (а), с приложенным электрическим полем (б)}
	\label{fig:mulener}
\end{figure}

Приложив внешнее электрическое поле к образцу, можно снизить потенциальный барьер, необходимый для испарения атома. Высота барьера Q(F) может быть описана следующей формулой \cite{Muller56}:

\begin{equation}
	\label{eq:equation2}
	Q(F) = Q_0 - \sqrt{\frac{n^3 e^3}{4\pi\epsilon_0}F}
\end{equation}

где $\epsilon_0$ - диэлектрическая проницаемость ьв вакууме, F - напряженность поля. Слагаемыми второго порядка, отвечающими поляризации атомов поверхности, как правило, пренебрегают. Поскольку основной задачей АЗТ является контролируемое по-атомное испарение, то необходимо контролировать скорость испарения атомов. Вероятность испарения может считаться термически активированным процессом, соответственно она может быть описана законом Аррениуса:

\begin{equation}
	\label{eq:equation3}
	P_{evap} \propto \exp(-\frac{Q(F)}{k_B T})
\end{equation}

где Q(F) - высота энергетического барьера \cref{eq:equation2}, $k_B$ - постоянная Больцмана, T - температура образца в Кельвинах. Скорость испарения определяется как:

\begin{equation}
	\label{eq:equation4}
	\Phi_{evap} = \nu_0\exp(-\frac{Q(F)}{k_B T})
\end{equation}

где $\nu_0$ - характерная частота колебаний атомов в направлении по нормали к поверхности образца. Данная зависимость подтверждена экспериментально \cite{Kellogg81,Kellogg84}, на Рисунке \cref{fig:evapspeed} представлены экспериментальные зависимости скорости испарения от приложенного электрического поля (в относительных величинах). Также наблюдались отклонения от данной зависимости \cite{Gomer84,Wada84}, которые характерны для температур менее 40 К и для ионов малой массы.

\begin{figure}[ht]
	\centerfloat{
		\includegraphics[scale=0.8]{evapspeed}
	}
	\caption{Зависимость скорости испарения от приложенного электрического поля для различных металлов в относительных единицах\cite{Tsong78}}
	\label{fig:evapspeed}
\end{figure} 

Значение поля, при котором величина энергетического барьера снижается до нуля, как правило, называется полем испарения $F_{evap}$. Учитывая \cref{eq:equation1} и \cref{eq:equation2}, это поле может быть вычислено для различных зарядовых состояний с помощью следующей формулы: 

\begin{equation}
	\label{eq:equation5}
	\Phi_{evap} = \frac{4\pi\epsilon_0}{n^3 e^3}(\Lambda + \sum_{1}^{n} I_n - n\phi_e)^2
\end{equation}

Результаты расчетов поля испарения для различных зарядностей ионов, на основе этого выражения, могут значительно различаться. Например, поле испарения для вольфрама принимает значения 102, 57, 52 и 62 В/нм для зарядностей +1, +2, +3 и +4, соответственно. Значения поля испарения для большинства элементов для наиболее распространенных зарядностей можно найти в справочной литературе. В работе Брандона \cite{Brandon65} было сделано предположение о том, что поле испарения определяется наименьшим из значений полей испарения для разных зарядностей иона. Данное предположение нашло хорошее подтверждение в проведенных экспериментах \cite{Tsong782}. Для большинства металлов поле испарения лежит в диапазон от 10 до 60 В/нм.
Поскольку на полевое испарение оказывает влияние температура и приложенное поле, то существует теоретически бесконечное число сочетаний значений поля и температуры, отвечающие одной и той же скорости испарения. В первом приближении высоту энергетического барьера рассматривают как величину, линейно меняющуюся весте с линейным изменением поля:
 
\begin{equation}
	\label{eq:equation6}
	Q(F) = Q_0(1 - \sqrt{1 - \sqrt{\frac{F}{F_{evap}}}}) \approx Q_0 (1 - \frac{F}{F_{evap}})
\end{equation}

Используя данное упрощенное выражение вместе с формулой \cref{eq:equation4} можно получить зависимость электрического поля необходимого для поддержания определенной скорости испарения в зависимости от температуры:

\begin{equation}
	\label{eq:equation7}
	\frac{F}{F_{evap}} \approx 1 + \ln{\frac{\Phi_{evap}}{\nu_0}\frac{k_B T}{Q_0}}
\end{equation}

Это простое выражение хорошо согласуется с экспериментальными наблюдениями различных авторов \cite{Kellogg84,Kellogg81,Wada84,Kellogg80,Vurpillot06}. Оно проиллюстрировано на Рисунке  \cref{fig:field_temp}, где поле, необходимое для испарения образца из чистого вольфрама регистрировалось как функция температуры. Соответствующее фактическое значение поля для заданной температуры обычно называют эффективным полем испарения.

\begin{figure}[ht]
	\centerfloat{
		\includegraphics[scale=0.8]{field_temp}
	}
	\caption{Зависимость поля от температуры при постоянной скорости детектирования \cite{Vurpillot06}}
	\label{fig:field_temp}
\end{figure} 

Как правило, такие кривые (как на Рисунке \cref{fig:field_temp}) используются для калибровки параметров сбора данных, то есть определения оптимальной температуры, скорости испарения и т.д. Вада, измеряя такие калибровочные кривые для чистых металлов, подчеркнул, что разные элементы будут иметь разные кривые \cite{Wada84}. При этом в сплавах будет иметь место сложная комбинация полей испарения каждого элемента в зависимости от химического состава и типов связей между элементами.



\section{Время-пролетная масс-спектрометрия в атомно-зондовой томографии}\label{sec:ch1/sec2}

Начиная с  1967 года, когда был собран первый атомно-зондовый томограф Мюллером, Паницем и МакЛайном в Государственном университете Пенсильвании \cite{Muller68} время-пролетная масс-спектрометрия была неотъемлемой частью атомно-зондовых томографов. Простые предположения и уравнения, предложенные в то время, до сих пор регулярно используются в современных установках АЗТ для определения химической природы испаряемых ионов. Конечные результаты, такие как состав и трехмерные координаты атомов, зависят от качества обработки масс-спектра.
Качество масс-спектра зависит не только от физики полевого испарения, но и от конструкции АЗТ, условий сбора данных, характера обработки данных. В АЗТ атомы испаряются с помощью электрического поля, они ионизируются и ускоряются в электрическом поле, окружающем поверхность образца (см. общие принципы работы \cref{sec:ch1/sec4}). Некоторые из этих ионов собираются детектором, используемым для определения времени и места удара. Время полета иона – это время, необходимое для полета от поверхности образца до детектора, которое называется длиной полета L (см. Рисунок \cref{fig:time_flight}). 

\begin{figure}[ht]
	\centerfloat{
		\includegraphics[scale=0.8]{time_flight}
	}
	\caption{Схема измерения времени пролета иона в АЗТ}
	\label{fig:time_flight}
\end{figure} 

Время пролета можно измерить только в том случае, если испарение инициируется напряжением или лазерным импульсом. Тогда, если предположить, что ионы не имеют начальной скорости и имеют постоянную скорость во время полета и что вся их энергия превращается в кинетическую энергию, то отношение массы к заряду M ионов записывается как:

\begin{equation}
	\label{eq:equation8}
	M = \frac{m}{n} = 2eV(\frac{t_f}{L})^2 = kV(\frac{t_f}{L})^2
\end{equation}

где m — масса иона, n — заряд, V — полное напряжение, приложенное к образцу, e — элементарный заряд электрона, L — длина полета и $t_f$ — время полета. Масса иона часто выражается в относительных единицах массы (а.е.м.). В современных установках АЗТ типичная длина полета составляет от 10 до 50 см, что дает характерное время полета в диапазоне около 1–5 мкс при напряжении в диапазоне 1–10 кВ.
Масс-спектр представляет собой гистограмму, которая дает распределение отношения массы к заряду собранных ионов. Как показано на Рисунке. \cref{fig:mass_spectr}, обычно присутствуют пики, соответствующие конкретным ионам. Разрешение масс-спектра определяется .

\begin{figure}[ht]
	\centerfloat{
		\includegraphics[scale=0.5]{mass_spectr}
	}
	\caption{Пример масс-спектра атомно-зондовых данных, полученный в работе ПППП}
	\label{fig:mass_spectr}
\end{figure} 

Идентификация пиков в масс-спектрах зависит как от предварительного знания химии анализируемого материала, так и от естественного содержания изотопов. Например, на Рисунке. \cref{fig:mass_spectr} идентифицированы различные изотопы Мо. Такая идентификация важна, поскольку она позволяет провести точную калибровку масс-спектрометра. Важно отметить, что одной из особенностей метода АЗТ является необходимость уточнения масс-спектра для каждого эксперимента. Качество масс-спектра оценивают по разрешению по массе $\frac{M}{\Delta M_x}$, где $\Delta M_x$ — ширина пика на масс-спектре при x процентах от максимальной высоты пика. Разрешение по массе обычно измеряется на уровне 50 $\%$ (полная ширина на половине высоты (FWHM)), 10 $\%$ и 1 $\%$ от максимума пика. Разрешение по массе зависит от геометрических факторов (длина и траектория полета) и точности измерений (времени пролета и напряжения). Оптимизация разрешения по массе может осуществляться с помощью аппаратного обеспечения (например, рефлектронов) или применении специальных алгоритмов оптимизации при обработке данных \cite{Shutov19}.


\section{Проекционный принцип восстановления координат атомов}\label{sec:ch1/sec3}

Как уже отмечалось, атомно-зондовая томография заимствовала принцип проекционного увеличения у автоионной микроскопии. Испаренные атомы «проецируются» на экран детектора использую расходящиеся линии электрического поля от кончика образца-иглы, что обеспечивает впечатляющее увеличение на детекторе, расположенном от образца некотором расстоянии. Область на образце порядка 100 нм легко проецируется на детектор с радиусом в 80-120 мм, что гарантирует разрешение, близкое к атомарному \cite{Cadel09}. Используемый принцип вносит требования к условиям испарения. Для того, чтобы «зафиксировать» атомы около их равновесных положений необходимо охлаждать изучаемый образец до криогенных температур. На практике используется диапазон от 15 до 80 К. Также желание детектировать, по возможности, каждый атом требует поддержания сверхвысокого вакуума не менее $10^{-9}$ Торр.

Для того чтобы иметь возможность восстанавливать положение атома в образце необходимо измерять его координаты прилета на детектор. На сегодняшний момент наиболее подходящими являются детекторы на основе линий задержки, также использующие микроканальные пластины для увеличения сигнала от прилетевшего иона \cite{DaCosta05,Jagutzki05}. Одним из важнейших параметров МКП является коэффициент открытой поверхности, который характеризует процент площади отверстий каналов к общей площади пластины. В настоящий момент у японского производителя МКП «Hamamatsu Photonics» \cite{Hamamatsu} наилучшим коэффициентом является 90 $\%$. Этот параметр напрямую влияет на полноту и качество получаемой информации об образце.

Базовый метод заключается в восстановлении X, Y и Z координат атомов в образце по получаемым значениям $X_d$ и $Y_d$ – координатам атомов зарегистрированных на детекторе и N – номеру события детектирования атома. Первым этапом рассматриваемого алгоритма является восстановление латеральных координат атомов в образце, то есть в плоскости детектора (X, Y). Этого можно достичь путем проецирования координат, полученных с детектора на область внутри иглы по формуле \cref{eq:equation9}:

\begin{equation}
	\label{eq:equation9}
	X = \frac{X_d}{\eta}; Y = \frac{Y_d}{\eta}
\end{equation} 

где $X_d$ и $Y_d$ - координаты события на детекторе, а $\eta$ рассчитывается по формуле:

\begin{equation}
	\label{eq:equation10}
	\eta = \frac{L}{\xi r_i}
\end{equation}

\begin{equation}
	\label{eq:equation11}
	r_i = \frac{U}{k_f E}
\end{equation}

где $\xi$ - изображающий параметр, $r_i$ - радиус поверхности, с которой был испарен атом. После проведения этих вычислений необходимо получить координату Z. Изначально предполагается, что процесс испарения происходит плавно, атом за атомом, слой за слоем. Таким образом, зная о приблизительном числе атомов, находящихся в одном слое, можно восстановить глубину, с которой испарился атом. Наиболее простой и широко распространенный метод расчета был предложен Басом \cite{Bas95}. Он предлагает это делать в несколько этапов следующим образом. Сначала рассчитать элементарный сдвиг по $z_i$ по формуле \cref{eq:equation12}, а затем провести суммирование по i, для получения полного смещения i-ого атома по Z координате.

\begin{equation}
	\label{eq:equation12}
	z_i = \frac{\Omega_i}{(V_i)^2} \left(\frac{(\delta k F)^2}{\zeta d_x d_y \xi^2}\right)
\end{equation}

где  $\Omega_i$ – объем приходящийся на i-й ион, F – напряженность поля необходимая для испарения, $V_i$  - потенциал в момент испарения, $\zeta$ – эффективность детектора, $\xi$ – изображающий параметр, $d_x d_y$ – координаты частицы на детекторе.


\section{Общие принципы работы атомно-зондового томографа. Современные установки АЗТ}\label{sec:ch1/sec4}



\section{Влияние параметров исследования на качество данных. Калибровка атомно-зондового томографа}\label{sec:ch1/sec5}

Здесь что-то о влиянии параметров

