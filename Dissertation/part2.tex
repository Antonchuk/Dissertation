\chapter{Разработка  Прототипа атомно-зондового томографа с лазерным испарением}\label{ch:ch2}

Разработки атомно-зондовой томографии в ИТЭФ (Россия) получили развитие в связи с модернизацией Центра атомно-масштабных и ядерно-физических микроскопических исследований конденсированных сред для получения разносторонней информации о наномасштабном состоянии различных материалов КАМИКС, включенного в перечень уникальных установок уникальных ядерно-физических установок, необходимых для осуществления национальным исследовательским центром «Курчатовский институт» своей деятельности (Распоряжение от 30 декабря 2009 г. №2125-р). Для расширения спектра исследований, проводящихся в ИТЭФ, в 2011 г. было принято решение о начале разработки стенда атомно-зондовой томографии нового поколения. В 2015 г. в ИТЭФ был запущен томографический атомный зонд с лазерным испарением и прямопролетной схемой конфигурации образца и детектора (см., Рис. 7, 8) \cite{scbibAPPLE}. Установка получила сокращенное название ПАЗЛ-3D – Прототип Атомного Зонда с Лазерным испарением. 

\section{Общая схема ПАЗЛ-3D}\label{sec:ch2/sec1}

В качестве детектирующей системы выбран позиционно чувствительный DLD детектор, позволяющий существенно повысить скорость сбора данных более чем на порядок по сравнению с имеющейся в лаборатории установкой ECOTAP. Он также позволяет с высокой точностью детектировать мультисобытия (одновременное попадание нескольких частиц на детектор).

картинка схема

картинка фото


Прямопролетная геометрия позволяет избежать сложностей с настройкой дополнительных отклоняющих ионы систем (ссылка), а также позволяет применять стандартные алгоритмы восстановления данных [ссылка]. В качестве базового варианта был выбран детектор производства RoentDek GmbH DLD120 с эффективным диаметром 120 мм, так как данная немецкая фирма производит наиболее современные позиционно-чувствительные детекторы на основе линий задержки для детектирования ионов. Детектирующая система состоит из сборки микроканальных пластин (МКП) и системы анодов, усилителя сигнала с детектора и аналого-цифрового преобразователя (АЦП) для оцифровки и передачи данных на компьютер. Оцифровка проводится с частотой дискретизации 5 ГГц для сигнала с МКП и с частотой 1 ГГц для сигналов с анодов. Принцип детектирования основан на линиях задержки и состоит в следующем. [ссылка??]. Ускоренный в поле вблизи образца ион попадает в канал МКП, порождает облако электронов, далее электроны попадают на систему анодов, наводя в проволоке анодов ЭДС. Затем сигнал усиливается в модуле усиления и передается на АЦП. Координаты прилета частиц на детекторе определяются с точностью менее 100 мкм, что позволяет обеспечить пространственное разрешение прототипа, близкое к атомарному (области образца с радиусом 50 нм ставится в соответствие область детектора с радиусом 60 мм). Эффективная площадь детектора вместе с длиной пролета ионов 183 мм позволяют достичь угла сбора данных 32, что сопоставимо с аналогичными установками [ссылка]. При проведении атомно-зондового исследования необходимо обеспечивать сверхвысокий вакуум в анализационном объеме. Схема расположения вакуумных объемов представлена на Рис. 9.

картинка схема

Откачка вакуумных объемов производится с помощью двух и трех ступенчатых систем насосов фирмы Pfeiffer Vacuum для загрузочного и анализационного объемов. Первой ступенью для обоих объемов выступает сухой спиральный форвакуумный насос. Он обеспечивает разряжение порядка 10-3 Торр. Далее установлены турбомолекулярные насосы, один на загрузочной камере и два на анализационном объеме. Давление в камерах измеряется с помощью ионизационных вакуумметров с горячим катодом. В анализационном объеме давление составляет 5.0 × 10-10 Торр, в загрузочном объеме – 4.0 × 10-9 Торр.

Для испарения атомов (ионов) на образец подается постоянное напряжение с помощью источника высоковольтного напряжения марки FuG Electronik GmbH (до 13 кВ). В качестве импульсного источника испарения используется лазерная система TEAT-25ST производства Авеста-Проект г. Москва. Данная система позволяет генерировать импульсы длительностью 60 или 300 фемтосекунд (в зависимости от настройки). Энергия импульса лежит в диапазоне от 0.1 до 250 мкДж. Основная длина волны составляет 1030 нанометров, из которой генерируются три гармоники для испарения образца: 515, 343 и 257 нанометров. Данные длины волн позволяют исследовать как металлы и полупроводники [Inoue K., Yano F., Nishida A., Takamizawa H., Tsunomura T., Nagai Y., Hasegawa M. // Ultramicroscopy. 2009. V. 109. P. 1479. DOI: 10.1016/j.ultramic.2009.08.002], так и некоторые диэлектрики [Gault B., Menand A., de Geuser F., Deconihout B., Danoix R. // Appl. Phys. Lett. 2006. V. 88. P. 14101. DOI: 10.1063/1.2186394]. Частота работы лазера составляет 50 кГц. Лазерное излучение фокусируется с помощью системы линз на кончик образца-иглы. Луч лазера заводится в камеру исследования с помощью системы зеркал и шаговых двигателей, обеспечивающей точность позиционирования менее 2 мкм. Введение лазерного луча в вакуумный объем проводится через специальное кварцевое окно с коэффициентом пропускания более 99% для любой из 3-х гармоник. Также лазерная система генерирует синхроимпульс для измерения времени пролета частиц, который далее подается на вход модуля АЦП в детектирующей системе.

\section{Вакуумная и криогенная системы}\label{sec:ch2/sec2}

\section{Фемтосекундная лазерная система испарения}\label{sec:ch2/sec3}

\section{Детектирующая система}\label{sec:ch2/sec4}

\section{Программное обеспечение для управления сбором атомно-зондовых данных}\label{sec:ch2/sec5}



%\FloatBarrier