


\chapter{Расчет разрешения по массе для различных длин пролета ионов для атомно-зондового томографа}\label{app:A}


\begin{landscape}
\chapter{Параметры исследования сплава Al-3.5Cu-0.2Mn-0.1S}\label{app:B}


\begin{longtable}{|p{6cm}|c|c|c|c|c|c|c|c|c|}
	\caption{Метрики качества атомно-зондовых данных}\\
	\hline
	\textbf{Номер набора данных} & \textbf{1} & \textbf{2} & \textbf{3} & \textbf{4} & \textbf{5} & \textbf{6} & \textbf{7}& \textbf{8} & \textbf{9}\\
	\hline
	\endfirsthead
	\multicolumn{10}{c}%
	{\tablename\ \thetable\ -- \textit{Начало на предыдущей странице}} \\
	\hline
	\textbf{Номер набора данных} & \textbf{1} & \textbf{2} & \textbf{3} & \textbf{4} & \textbf{5} & \textbf{6} & \textbf{7}& \textbf{8} & \textbf{9} \\
	\hline
	\endhead
	\hline \multicolumn{10}{r}{\textit{Продолжение на следующей странице}} \\
	\endfoot
	\hline
	\endlastfoot
	Мощность, отн. ед. & 400 & 500 & 600 & 700 & 800 & 900 & 1000 & 500 & 600  \\ \hline
	Однократные события, \% & 97.06 & 97.79 & 98.41 & 97.14 & 94.48 & 95.56 & 96.48 & 94.93 & 97.36              \\ \hline
	Доля мультисобытий Cu, \% & 4.12 & 3.66 & 4.36 & 2.72 & 2.89 & 2.94 & 3.11 & 2.78 & 3.24     \\ \hline
	Поток событий, атомов/1000 воздействий    & 0.0045 & 0.0045 & 0.0035 & 0.0055 & 0.0085 & 0.008 & 0.007 & 0.004 & 0.0055      \\ \hline
	Напряжение, кВ  & 5.0-5.7 & 4.2-4.9 & 3.2-4.2 & 5.8-7.1 & 7.4-7.8 & 7.8-8.5 & 8.5-9.0 & 9.6-9.9 & 5.5-5.8        \\ \hline
	Шум, \%         & 14 & 8.5 & 6 & 20 & 15 & 17 & 21 & 42 & 11       \\ \hline
	Шум на промежутке 10-11 а.е.м., отн. ед.    & 100 & 40 & 18 & 54 & 49 & 56 & 61 & 213 & 45      \\ \hline
	Шум на промежутке 40-41 а.е.м., отн. ед.    & 135 & 65 & 24 & 100 & 124 & 156 & 210 & 495 & 83   \\ \hline
	Cu concentration, \%    & 0.53 & 0.70 & 0.89 & 0.77 & 0.88 & 0.85 & 0.91 & 0.59 & 0.75   \\ \hline
	Sn concentration, \%    & - & 0.02 & 0.01 & 0.01 & - & <0.01 & <0.01 & - & -    \\ \hline
	Mn concentration, \%    & - & 0.006 & 0.01 & 0.006 & - & - & 0.005 & - & -    \\ \hline
	Соотношение Al$^+$/Al$^{++}$, отн. ед., \%    & 78 & 170 & 386 & 210 & 182 & 230 & 289 & 73 & 168    \\ \hline
	Число атомов (0-100 а.е.м.)    & \thead{514 748} & \thead{558 408} & \thead{591 629} & \thead{4 538 007} & \thead{2 648 840} & \thead{5 898 660} & \thead{7 653 641} & \thead{1 899 266} & \thead{581 324}      \\ \hline
\end{longtable}

\clearpage

\begin{longtable} {| p{6cm} | c | c | c | c | c | c | c | c | c |}
	\caption{Метрики качества атомно-зондовых данных}\\
	\hline
	\textbf{Номер набора данных} & \textbf{1} & \textbf{2} & \textbf{3} & \textbf{4} & \textbf{5} & \textbf{6} & \textbf{7}& \textbf{8} & \textbf{9}\\
	\hline
	\endfirsthead
	\multicolumn{10}{c}%
	{\tablename\ \thetable\ -- \textit{Начало на предыдущей странице}} \\
	\hline
	\textbf{Номер набора данных} & \textbf{1} & \textbf{2} & \textbf{3} & \textbf{4} & \textbf{5} & \textbf{6} & \textbf{7}& \textbf{8} & \textbf{9} \\
	\hline
	\endhead
	\hline \multicolumn{10}{r}{\textit{Продолжение на следующей странице}} \\
	\endfoot
	\hline
	\endlastfoot
		Мощность, отн. ед. & 80 & 100 & 110 & 130 & 130 & 110 & 150 & 150 & 170  \\ \hline
		Однократные события, \% & 97.35 & 96.70 & 96.43 & 97.45 & 96.13 & 97.26 & 97.19 & 96.78 & 97.29              \\ \hline
		Доля мультисобытий Cu, \% & 3.13 & 3.07 & 2.66 & 3.80 & 2.35 & 3.43 & 3.07 & 3.02 & 3.25               \\ \hline
		Шум, \%         & 14 & 8.5 & 6 & 20 & 15 & 17 & 21 & 42 & 11               \\ \hline
		Шум на промежутке 10-11 а.е.м., отн. ед.    & 100 & 40 & 18 & 54 & 49 & 56 & 61 & 213 & 45      \\ \hline
		Шум на промежутке 40-41 а.е.м., отн. ед.    & 135 & 65 & 24 & 100 & 124 & 156 & 210 & 495 & 83      \\ \hline
		Cu concentration, \%    & 0.53 & 0.70 & 0.89 & 0.77 & 0.88 & 0.85 & 0.91 & 0.59 & 0.75      \\ \hline
		Sn concentration, \%    & - & 0.02 & 0.01 & 0.01 & - & <0.01 & <0.01 & - & -      \\ \hline
		Mn concentration, \%    & - & 0.006 & 0.01 & 0.006 & - & - & 0.005 & - & -      \\ \hline
		Соотношение Al$^+$/Al$^{++}$, отн. ед., \%    & 78 & 170 & 386 & 210 & 182 & 230 & 289 & 73 & 168      \\ \hline
		Число атомов (0-100 а.е.м.)    & \thead{514 748} & \thead{558 408} & \thead{591 629} & \thead{4 538 007} & \thead{2 648 840} & \thead{5 898 660} & \thead{7 653 641} & \thead{1 899 266} & \thead{581 324}      \\ \hline
\end{longtable}
\end{landscape}

\chapter{Пример листинга кода}\label{app:C}
Для крупных листингов есть два способа. Первый красивый, но в нём могут быть
проблемы с поддержкой кириллицы (у вас может встречаться в~комментариях
и~печатаемых сообщениях), он представлен на листинге~\cref{lst:hwbeauty}.
\begin{ListingEnv}[!h]% настройки floating аналогичны окружению figure
    \captiondelim{ } % разделитель идентификатора с номером от наименования
    \caption{Программа ,,Hello, world`` на \protect\cpp}\label{lst:hwbeauty}
    % окружение учитывает пробелы и табуляции и применяет их в сответсвии с настройками
    \begin{lstlisting}[language={[ISO]C++}]
	#include <iostream>
	using namespace std;

	int main() //кириллица в комментариях при xelatex и lualatex имеет проблемы с пробелами
	{
		cout << "Hello, world" << endl; //latin letters in commentaries
		system("pause");
		return 0;
	}
    \end{lstlisting}
\end{ListingEnv}%
Второй не~такой красивый, но без ограничений (см.~листинг~\cref{lst:hwplain}).
\begin{ListingEnv}[!h]
    \captiondelim{ } % разделитель идентификатора с номером от наименования
    \caption{Программа ,,Hello, world`` без подсветки}\label{lst:hwplain}
    \begin{Verb}

        #include <iostream>
        using namespace std;

        int main() //кириллица в комментариях
        {
            cout << "Привет, мир" << endl;
        }
    \end{Verb}
\end{ListingEnv}



\clearpage
\refstepcounter{chapter}