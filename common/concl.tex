%% Согласно ГОСТ Р 7.0.11-2011:
%% 5.3.3 В заключении диссертации излагают итоги выполненного исследования, рекомендации, перспективы дальнейшей разработки темы.
%% 9.2.3 В заключении автореферата диссертации излагают итоги данного исследования, рекомендации и перспективы дальнейшей разработки темы.
\begin{enumerate}[beginpenalty=10000]
	\item Впервые в России создана установка атомно-зондовой томографии с фемтосекундным лазерным испарением позволяющая проводить атомно-зондовые исследования широкого спектра материалов  и получать информацию о трехмерном распределении элементов в объеме образца с разрешением, близким к атомному. Пространственное разрешение установки составляет не хуже 4 \r{A}. Разрешение по массе на полувысоте на ПАЗЛ-3D в среднем находится в диапазоне от 200 до 600~отн.~ед. (в зависимости от материала и условий исследований).
	\item Разработана оригинальная методика коррекции атомно-зондовых данных по атомной плотности материала для компенсации ошибки восстановления трехмерных координат. Данная методика обеспечивает восстановление 3D координат атомов точнее на 50~\%, чем стандартные алгоритмы обработки для длины исследуемого объема более 300 нм.   
	\item Разработана методика контроля условий испарения для разных атомно-зондовых томографах с использованием метрики соотношения зарядностей одно- и двухзарядных пиков основного химического элемента материала образца. Сравнение результатов исследований алюминиевого сплава и стали показало, что отклонение ключевых характеристик находится в пределах погрешности или допустимых отклонений для данных материалов. Данная методика позволит улучшит повторяемость результатов исследований.
	\item Апробирована методика сопоставления и сравнения условий испарения на разных установках атомно-зондовой томографии с использованием метрики соотношения зарядностей основного химического элемента материал. Продемонстрировала возможность проводить и сопоставлять АЗТ данные, полученные на разных установках АЗТ.
	\item В результате проведения исследований набора сплавов изучены возможности установки по исследованию материалов с нано-размерными особенностями. Показаны способы обработки и представления результатов исследований, такие как: поиск кластеров, масс-спектр, атомные карты, профили концентраций, анализ состава материала, изо-концентрационные поверхности.  Проведены демонстрационные исследования состава и структуры ряда сплавов с помощью атомно-зондовой томографии:
	\begin{itemize}
		\item Получены атомные карты химических элементов для сплава Al-Mg-Si. Измерены плотность и характерные размеры включений Mg-Si. Получен состав нано-размерных включений
		\item Исследованы сегрегации атомов углерода(кластеры) и крупные карбидные частицы в образцах среднеуглеродистой стали после различных температур отпуска.
		\item Подтверждено наличие кластеров углерода в экономнолегированной стали при высокотемпературном отпуске
		\item О сплаве алюминия (ожидает согласования с МИСиС)
	\end{itemize}
\end{enumerate}

