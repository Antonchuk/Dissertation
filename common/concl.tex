%% Согласно ГОСТ Р 7.0.11-2011:
%% 5.3.3 В заключении диссертации излагают итоги выполненного исследования, рекомендации, перспективы дальнейшей разработки темы.
%% 9.2.3 В заключении автореферата диссертации излагают итоги данного исследования, рекомендации и перспективы дальнейшей разработки темы.
\begin{enumerate}[beginpenalty=10000]
	\item Впервые в России создана установка атомно-зондовой томографии с фемтосекундным лазерным испарением позволяющая проводить атомно-зондовые исследования широкого спектра материалов  и получать информацию о трехмерном распределении элементов в объеме образца с разрешением, близким к атомному.
	\item Разработана оригинальная методика коррекции атомно-зондовых данных по атомной плотности материала для компенсации ошибки восстановления трехмерных координат, обеспечивающая восстановление 3D координат атомов точнее, чем стандартные алгоритмы обработки.     
	\item Разработана методика контроля условий испарения для разных атомно-зондовых томографах с использованием метрики соотношения зарядностей одно- и двухзарядных пиков алюминия для алюминиевых сплавов, тем самым, повышая качество и воспроизводимость результатов исследований.
	\item Апробирована методика сопоставления и сравнения условий испарения на разных установках атомно-зондовой томографии с использованием метрики соотношения зарядностей основного химического элемента материал. Продемонстрировала возможность проводить и сопоставлять АЗТ данные, полученные на разных установках АЗТ.
	\item Проведены демонстрационные исследования состава и структуры ряда сплавов с помощью атомно-зондовой томографии:
	\begin{itemize}
		\item Получены атомные карты химических элементов для сплава Al-Mg-Si. Измерены плотность и характерные размеры включений Mg-Si. Получен состав нано-размерных включений
		\item Исследованы сегрегации атомов углерода(кластеры) и крупные карбидные частицы в образцах среднеуглеродистой стали после различных температур отпуска.
		\item Подтверждено наличие кластеров углерода в экономнолегированной стали при высокотемпературном отпуске
		\item О сплаве алюминия (ожидает согласования с МИСиС)
	\end{itemize}
\end{enumerate}

