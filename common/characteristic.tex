
{\actuality} Разработка новых материалов для ядерной техники, микроэлектронике и других высокотехнологичных направления требует учета не только микроструктурных особенностей материала, но и информацию о атомном распределении химических элементов. Эти сведения составляют основу для описания радиационной стойкости, прочности и других макроскопических характеристик материалов. Кроме того, информация эволюции нано-размерных включений в материале применяется в интересах разработки и интерпретации новых эффектов на масштабе, близком к атомному.
Для контроля морфологии и химического состава наноразмерных объектов наилучшим образом подходит методика атомно-зондовой томографии. Практически все ведущие зарубежные ядерные центры оснащены такими приборами, включая аналитические центры крупных международных университетов (например, Окриджская национальная лаборатория (США), Оксфордский университет (Англия), Руанский университет (Франция), Институт технологий Карлсруэ (Германия), Университет Тохоку (Япония), Пекинский технологический университет (Китай) и др. [1]). Необходимо отметить, что атомно-зондовая томография – единственная методика позволяющая получить трехмерное изображение химической структуры материала с атомарным разрешением. На сегодняшний день для аналогичных целей используются иные методики анализа, например, электронная микроскопия (просвечивающая, сканирующая), малоугловое рассеяние нейтронов, позитронная аннигилляционная спектроскопия, Мёссбауэровская спектроскопия. При определенных условиях указанные методики могут обеспечивать практически атомарное разрешение (просвечивающая электронная микроскопия высоко разрешения [2]) или иметь высокое разрешение по массе (вторично-ионная масс-спектрометрия). Но ни одна из этих методик не может позволить получить истинно трехмерное отображение структуры материала с разрешением, близким к атомарному, одновременно с определением химической природы каждого зарегистрированного атома [3].


\ifsynopsis
Этот абзац появляется только в~автореферате.
Для формирования блоков, которые будут обрабатываться только в~автореферате,
заведена проверка условия \verb!\!\verb!ifsynopsis!.
Значение условия задаётся в~основном файле документа (\verb!synopsis.tex! для
автореферата).
\else
\fi

% {\progress}
% Этот раздел должен быть отдельным структурным элементом по
% ГОСТ, но он, как правило, включается в описание актуальности
% темы. Нужен он отдельным структурынм элемементом или нет ---
% смотрите другие диссертации вашего совета, скорее всего не нужен.

{\aim} данной диссертационной работы является разработка и запуск комплекса атомно-зондовой томографии с фемтосекундным лазерным испарением, в том числе: разработка методик проведения исследований материалов, проведение исследований полупроводящих материалов и конструкционных сталей.%; разработка собственного ПО

Для~достижения поставленной цели необходимо было решить следующие {\tasks}:
\begin{enumerate}[beginpenalty=10000] % https://tex.stackexchange.com/a/476052/104425
  \item Разработать и запустить установку атомно-зондовой томографии с фемто-секундным лазерным испарением с пространственным разрешением не хуже 4 ангстрем и разрешением по массе не хуже 500.
  \item Продемонстрировать возможность исследования сплавов с нано-размерными особенностями
  \item Разработать методику проведения исследований феррито-маретенситных сталей.
  \item Разработать методику проведения исследования полупроводящих материалов.
  \item Разработать методы коррекции для новой установки базового метода восстановления атомно-зондовых данных.  
\end{enumerate}


{\novelty}
Впервые в России разработана и создана установка атомно-зондовой томографии. Проведены исследования материалов с нано-размерными особенностями. Разработаны и апробированы оригинальные методики проведения исследований. Показаны возможности установки по исследованию различных материалов.
%\begin{enumerate}[beginpenalty=10000] % https://tex.stackexchange.com/a/476052/104425
  %\item Впервые \ldots
  %\item Впервые \ldots
  %\item Было выполнено оригинальное исследование \ldots
%\end{enumerate}

{\influence} Созданная установка АЗТ позволила провести ряд уникальных исследований структуры и состава материалов в рамках исследования радиационной стойкости материалов ядерной техники. Полученный опыт при разработке установки позволит модернизировать имеющиеся атомно-зондовые томографы с полевым испарением и разрабатывать новые томографы в интересах научных институтов атомной отрасли России. Были проведены исследования для прогноза радиационной стойкости перспективных ДУО сталей. Данные работы могут иметь практическое значение для научных организаций: ВНИИНМ им. Бочвара, ЦНИИКМ "Прометей", НИТУ МИСиС.
%{\methods} \ldots

{\defpositions}
\begin{enumerate}[beginpenalty=10000] % https://tex.stackexchange.com/a/476052/104425
  \item Первая в России установка атомно-зондовой томографии с фемто-секундным лазерным испарением с пространственным разрешением не хуже 4 Ангстрем и разрешением по массе не хуже 500 отн. ед.
  \item Результаты исследования структуры материалов с помощью атомно-зондовой томографии:
  	\begin{itemize}
  		\item Получены атомные карты химических элементов для сплава Al-Mg-Si. Измерены плотность и характерные размеры включений Mg-Si. Получен состав нано-размерных включений
  		\item Продемонстрировано, что HighPressureTorsion приводит к сегрегации атомов Ca и La вдоль границ зерен и субзерен. Определены средние размеры составы частиц Ca-La. Проведена оценка растворимости лантана в матрице для сплава Al-Ca-La
  		\item 
  	\end{itemize}
  \item Оригинальные методики проведения исследований
  \item Оригинальная методика коррекции данных
\end{enumerate}

{\reliability} полученных результатов обеспечивается \ldots \ Результаты находятся в соответствии с результатами, полученными другими авторами.


{\probation}
Основные результаты работы докладывались на российских и международных конференциях: 

{\contribution} Диссертант Лукьянчук А.А. внес решающий вклад в создание, эксплуатацию и проведение экспериментов на первом в России атомно-зондовом томографе с лазерным испарением – ПАЗЛ-3D. Диссертант лично разрабатывал схему установки и компоновку узлов; систему загрузки/~выгрузки образцов, систему охлаждения образца в анализационном объеме. Он непосредственно проводил пуско-наладочные работы, проводил эксперименты по исследованию различных материалов и обрабатывал полученные данные. Диссертант также участвовав в разработке ПО обработки АЗТ данных (KVANTM-3D). В процессе работы на установке диссертантом создана оригинальная методика коррекции восстановления 3Д данных и зарегистрировано НОУ-ХАУ по улучшению качества данных, получаемых на установке ПАЗЛ-3D. Диссертант непосредственно участвовал в подготовке всех публикаций по теме квалификационной работы.

\ifnumequal{\value{bibliosel}}{0}
{%%% Встроенная реализация с загрузкой файла через движок bibtex8. (При желании, внутри можно использовать обычные ссылки, наподобие `\cite{vakbib1,vakbib2}`).
    {\publications} Основные результаты по теме диссертации изложены
    в~XX~печатных изданиях,
    X из которых изданы в журналах, рекомендованных ВАК,
    X "--- в тезисах докладов.
}%
{%%% Реализация пакетом biblatex через движок biber
    \begin{refsection}[bl-author, bl-registered]
        % Это refsection=1.
        % Процитированные здесь работы:
        %  * подсчитываются, для автоматического составления фразы "Основные результаты ..."
        %  * попадают в авторскую библиографию, при usefootcite==0 и стиле `\insertbiblioauthor` или `\insertbiblioauthorgrouped`
        %  * нумеруются там в зависимости от порядка команд `\printbibliography` в этом разделе.
        %  * при использовании `\insertbiblioauthorgrouped`, порядок команд `\printbibliography` в нём должен быть тем же (см. biblio/biblatex.tex)
        %
        % Невидимый библиографический список для подсчёта количества публикаций:
        \printbibliography[heading=nobibheading, section=1, env=countauthorvak,          keyword=biblioauthorvak]%
        \printbibliography[heading=nobibheading, section=1, env=countauthorwos,          keyword=biblioauthorwos]%
        \printbibliography[heading=nobibheading, section=1, env=countauthorscopus,       keyword=biblioauthorscopus]%
        \printbibliography[heading=nobibheading, section=1, env=countauthorconf,         keyword=biblioauthorconf]%
        \printbibliography[heading=nobibheading, section=1, env=countauthorother,        keyword=biblioauthorother]%
        \printbibliography[heading=nobibheading, section=1, env=countregistered,         keyword=biblioregistered]%
        \printbibliography[heading=nobibheading, section=1, env=countauthorpatent,       keyword=biblioauthorpatent]%
        \printbibliography[heading=nobibheading, section=1, env=countauthorprogram,      keyword=biblioauthorprogram]%
        \printbibliography[heading=nobibheading, section=1, env=countauthor,             keyword=biblioauthor]%
        \printbibliography[heading=nobibheading, section=1, env=countauthorvakscopuswos, filter=vakscopuswos]%
        \printbibliography[heading=nobibheading, section=1, env=countauthorscopuswos,    filter=scopuswos]%
        %
        \nocite{*}%
        %
        {\publications} Основные результаты по теме диссертации изложены в~\arabic{citeauthor}~печатных изданиях,
        \arabic{citeauthorvak} из которых изданы в журналах, рекомендованных ВАК\sloppy%
        \ifnum \value{citeauthorscopuswos}>0%
            , \arabic{citeauthorscopuswos} "--- в~периодических научных журналах, индексируемых Web of~Science и Scopus\sloppy%
        \fi%
        \ifnum \value{citeauthorconf}>0%
            , \arabic{citeauthorconf} "--- в~тезисах докладов.
        \else%
            .
        \fi%
        \ifnum \value{citeregistered}=1%
            \ifnum \value{citeauthorpatent}=1%
                Зарегистрирован \arabic{citeauthorpatent} патент.
            \fi%
            \ifnum \value{citeauthorprogram}=1%
                Зарегистрирована \arabic{citeauthorprogram} программа для ЭВМ.
            \fi%
        \fi%
        \ifnum \value{citeregistered}>1%
            Зарегистрированы\ %
            \ifnum \value{citeauthorpatent}>0%
            \formbytotal{citeauthorpatent}{патент}{}{а}{}\sloppy%
            \ifnum \value{citeauthorprogram}=0 . \else \ и~\fi%
            \fi%
            \ifnum \value{citeauthorprogram}>0%
            \formbytotal{citeauthorprogram}{программ}{а}{ы}{} для ЭВМ.
            \fi%
        \fi%
        % К публикациям, в которых излагаются основные научные результаты диссертации на соискание учёной
        % степени, в рецензируемых изданиях приравниваются патенты на изобретения, патенты (свидетельства) на
        % полезную модель, патенты на промышленный образец, патенты на селекционные достижения, свидетельства
        % на программу для электронных вычислительных машин, базу данных, топологию интегральных микросхем,
        % зарегистрированные в установленном порядке.(в ред. Постановления Правительства РФ от 21.04.2016 N 335)
    \end{refsection}%
    \begin{refsection}[bl-author, bl-registered]
        % Это refsection=2.
        % Процитированные здесь работы:
        %  * попадают в авторскую библиографию, при usefootcite==0 и стиле `\insertbiblioauthorimportant`.
        %  * ни на что не влияют в противном случае
        %\nocite{vakbib2}%vak
        %\nocite{patbib1}%patent
        %\nocite{progbib1}%program
        %\nocite{bib1}%other
        %\nocite{confbib1}%conf
    \end{refsection}%
        %
        % Всё, что вне этих двух refsection, это refsection=0,
        %  * для диссертации - это нормальные ссылки, попадающие в обычную библиографию
        %  * для автореферата:
        %     * при usefootcite==0, ссылка корректно сработает только для источника из `external.bib`. Для своих работ --- напечатает "[0]" (и даже Warning не вылезет).
        %     * при usefootcite==1, ссылка сработает нормально. В авторской библиографии будут только процитированные в refsection=0 работы.
}



При использовании пакета \verb!biblatex! будут подсчитаны все работы, добавленные
в файл \verb!biblio/author.bib!. Для правильного подсчёта работ в~различных
системах цитирования требуется использовать поля:
\begin{itemize}
	\item \texttt{authorvak} если публикация индексирована ВАК,
	\item \texttt{authorscopus} если публикация индексирована Scopus,
	\item \texttt{authorwos} если публикация индексирована Web of Science,
	\item \texttt{authorconf} для докладов конференций,
	\item \texttt{authorpatent} для патентов,
	\item \texttt{authorprogram} для зарегистрированных программ для ЭВМ,
	\item \texttt{authorother} для других публикаций.
\end{itemize}
Для подсчёта используются счётчики:
\begin{itemize}
	\item \texttt{citeauthorvak} для работ, индексируемых ВАК,
	\item \texttt{citeauthorscopus} для работ, индексируемых Scopus,
	\item \texttt{citeauthorwos} для работ, индексируемых Web of Science,
	\item \texttt{citeauthorvakscopuswos} для работ, индексируемых одной из трёх баз,
	\item \texttt{citeauthorscopuswos} для работ, индексируемых Scopus или Web of~Science,
	\item \texttt{citeauthorconf} для докладов на конференциях,
	\item \texttt{citeauthorother} для остальных работ,
	\item \texttt{citeauthorpatent} для патентов,
	\item \texttt{citeauthorprogram} для зарегистрированных программ для ЭВМ,
	\item \texttt{citeauthor} для суммарного количества работ.
\end{itemize}

% Счётчик \texttt{citeexternal} используется для подсчёта процитированных публикаций;
% \texttt{citeregistered} "--- для подсчёта суммарного количества патентов и программ для ЭВМ.

Для добавления в список публикаций автора работ, которые не были процитированы в
автореферате, требуется их~перечислить с использованием команды \verb!\nocite! в
\verb!Synopsis/content.tex!.
