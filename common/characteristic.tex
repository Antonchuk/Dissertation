
{\actuality} Разработка новых материалов для ядерной техники, микроэлектронике и других высокотехнологичных направления требует учета не только микроструктурных особенностей материала, но и информацию о атомном распределении химических элементов. Эти сведения составляют основу для описания радиационной стойкости, прочности и других макроскопических характеристик материалов. Кроме того, информация эволюции нано-размерных включений в материале применяется в интересах разработки и интерпретации новых эффектов на масштабе, близком к атомному.
Для контроля морфологии и химического состава наноразмерных объектов наилучшим образом подходит методика атомно-зондовой томографии. Практически все ведущие зарубежные ядерные центры оснащены такими приборами, включая аналитические центры крупных международных университетов (например, Окриджская национальная лаборатория (США), Оксфордский университет (Англия), Руанский университет (Франция), Институт технологий Карлсруэ (Германия), Университет Тохоку (Япония), Пекинский технологический университет (Китай) и др. [1]). Необходимо отметить, что атомно-зондовая томография – единственная методика, позволяющая получить трехмерное изображение химической структуры материала с атомарным разрешением. На сегодняшний день для аналогичных целей используются иные методики анализа, например, электронная микроскопия (просвечивающая, сканирующая), малоугловое рассеяние нейтронов, позитронная аннигилляционная спектроскопия, Мёссбауэровская спектроскопия. При определенных условиях указанные методики могут обеспечивать практически атомарное разрешение (просвечивающая электронная микроскопия высоко разрешения [2]) или иметь высокое разрешение по массе (вторично-ионная масс-спектрометрия). Но ни одна из этих методик не может позволить получить истинно трехмерное отображение структуры материала с разрешением, близким к атомарному, одновременно с определением химической природы каждого зарегистрированного атома [3].


\ifsynopsis
Этот абзац появляется только в~автореферате.
Для формирования блоков, которые будут обрабатываться только в~автореферате,
заведена проверка условия \verb!\!\verb!ifsynopsis!.
Значение условия задаётся в~основном файле документа (\verb!synopsis.tex! для
автореферата).
\else
\fi

% {\progress}
% Этот раздел должен быть отдельным структурным элементом по
% ГОСТ, но он, как правило, включается в описание актуальности
% темы. Нужен он отдельным структурынм элемементом или нет ---
% смотрите другие диссертации вашего совета, скорее всего не нужен.

{\aim} данной диссертационной работы является разработка и запуск комплекса атомно-зондовой томографии с фемтосекундным лазерным испарением, в том числе: разработка методик проведения исследований материалов, проведение исследований полупроводящих материалов и конструкционных сталей.%; разработка собственного ПО

Для~достижения поставленной цели необходимо было решить следующие {\tasks}:
\begin{enumerate}[beginpenalty=10000] % https://tex.stackexchange.com/a/476052/104425
  \item Разработать и запустить установку атомно-зондовой томографии с фемто-секундным лазерным испарением с пространственным разрешением не хуже 4 ангстрем и разрешением по массе не хуже 500.
  \item Продемонстрировать возможность исследования сплавов с нано-размерными особенностями
  \item Разработать методику проведения исследований феррито-маретенситных сталей.
  \item Разработать методику проведения исследования полупроводящих материалов.
  \item Разработать методы коррекции для новой установки базового метода восстановления атомно-зондовых данных.  
\end{enumerate}


{\novelty}
Впервые в России разработана и создана установка атомно-зондовой томографии. Проведены исследования материалов с нано-размерными особенностями. Разработаны и апробированы оригинальные методики проведения исследований. Показаны возможности установки по исследованию различных материалов.

{\influence} Созданная установка АЗТ позволила провести ряд уникальных исследований структуры и состава материалов в рамках исследования радиационной стойкости материалов ядерной техники. Полученный опыт при разработке установки позволит модернизировать имеющиеся атомно-зондовые томографы с полевым испарением и разрабатывать новые томографы в интересах научных институтов атомной отрасли России. Были проведены исследования для прогноза радиационной стойкости перспективных ДУО сталей. Данные работы могут иметь практическое значение для научных организаций: ВНИИНМ им. Бочвара, ЦНИИКМ "Прометей", НИТУ МИСиС.
%{\methods} \ldots

{\defpositions}
\begin{enumerate}[beginpenalty=10000] % https://tex.stackexchange.com/a/476052/104425
  \item Первая в России установка атомно-зондовой томографии с фемто-секундным лазерным испарением с пространственным разрешением не хуже 4 Ангстрем и разрешением по массе не хуже 500 отн. ед.
  \item Результаты исследования структуры материалов с помощью атомно-зондовой томографии:
  	\begin{itemize}
  		\item Получены атомные карты химических элементов для сплава Al-Mg-Si. Измерены плотность и характерные размеры включений Mg-Si. Получен состав нано-размерных включений
  		\item Продемонстрировано, что HighPressureTorsion приводит к сегрегации атомов Ca и La вдоль границ зерен и субзерен. Определены средние размеры составы частиц Ca-La. Проведена оценка растворимости лантана в матрице для сплава Al-Ca-La
  		\item 
  	\end{itemize}
  \item Оригинальные методики проведения исследований
  \item Оригинальная методика коррекции данных
\end{enumerate}

{\reliability} полученных результатов обеспечивается \ldots \ Результаты находятся в соответствии с результатами, полученными другими авторами.


{\probation}
Основные результаты работы докладывались на российских и международных конференциях: 

{\contribution} Диссертант Лукьянчук А.А. внес решающий вклад в создание, эксплуатацию и проведение экспериментов на первом в России атомно-зондовом томографе с лазерным испарением – ПАЗЛ-3D. Диссертант лично разрабатывал схему установки и компоновку узлов; систему загрузки/~выгрузки образцов, систему охлаждения образца в анализационном объеме. Он непосредственно проводил пуско-наладочные работы, проводил эксперименты по исследованию различных материалов и обрабатывал полученные данные. Диссертант также участвовав в разработке ПО обработки АЗТ данных (KVANTM-3D). В процессе работы на установке диссертантом создана оригинальная методика коррекции восстановления 3Д данных и зарегистрировано НОУ-ХАУ по улучшению качества данных, получаемых на установке ПАЗЛ-3D. Диссертант непосредственно участвовал в подготовке всех публикаций по теме квалификационной работы.

