
{\actuality} Современный материаловедческие разработки в различных областях (ядерная техника, авиационно-космическая техника, микроэлектроника и другие высокотехнологичные направления) требуют учета не только микроструктурных особенностей материалов, но и деталей их нано-масщштабного или даже атомно-масштабного состояния. 

В качестве одного из наиболее продвинутых методов контроля морфологии и химического состава наноразмерных объектов в многокомпонентных материалах наилучшим образом подходит атомно-зондовая томография (АЗТ) (в англоязычной литературе atom probe tomography - APT). Многие ведущие зарубежные ядерные центры оснащены такими приборами, включая аналитические центры крупных международных университетов (например, Окриджская национальная лаборатория (США), Оксфордский университет (Англия), Руанский университет (Франция), Институт технологий Карлсруэ (Германия), Университет Тохоку (Япония), Пекинский технологический университет (Китай) и др. \cite{APTlist}). Необходимо отметить, что атомно-зондовая томография – единственная методика, позволяющая получить трехмерное изображение химической структуры материала с атомарным разрешением. На сегодняшний день для аналогичных целей используются иные методики анализа, например, электронная микроскопия (просвечивающая, сканирующая), малоугловое рассеяние нейтронов, позитронная аннигилляционная спектроскопия, Мёссбауэровская спектроскопия. При определенных условиях указанные методики могут обеспечивать практически атомарное разрешение (просвечивающая электронная микроскопия высоко разрешения) или иметь высокое разрешение по массе (вторично-ионная масс-спектрометрия). Но ни одна из этих методик не может позволить получить истинно трехмерное отображение структуры материала с разрешением, близким к атомарному, одновременно с определением химической природы каждого зарегистрированного атома \cite{GaultBOOK}. С помощью атомно-зондовой томографии можно исследовать широкий спектр материалов, например: материалы ядерных энергетических установок, материалы для газотурбинных двигателей, материалы и изделия полупроводниковой промышленности \cite{Ulfig23}, различные биологические материалы \cite{Lee21}. Методика атомно-зондовой томографии часто используется как комплементарный метод анализа наномасштабных объектов в сложно структурированных материалах, имеющих различные особенности на различных пространственных масштабах (зеренную структуру на микромасштабах, вторичные фазы на масштабах от десятков до сотен нанометров, и нанокластеры/предвыделения фаз с размерами единиц нанометров (см, например \cite{Rogozhkin24_nano}).

В России, до недавних пор, инструментальная база для атомно-зондовых исследований была представлена всего двумя установками. Первая установка ECOTAP была запущена в 2003 году в Институте Теоретической и Экспериментальной физики (сейчас является частью НИЦ "Курчатовский институт"). Это одна из первых коммерческих установок АЗТ, она имела электрическое импульсное испарение и рефлектрон для повышения разрешения по массе. Вторая - LEAP 400 HR запущена в 2015 г. в НИЦ "Курчатовский институт". Она имеет более совершенную электрическую систему испарения типа "локальный электрод". Этих установок явно не достаточно для современных отечественных разработок в области материаловедения. Более того, они не отвечают широкому кругу современных материаловедческих разработок, которые выходят за пределы металлических систем, для которых подходят АЗТ установки с электрическим испарением. Учитывая имеющиеся ограничения поставок такого рода установок, создание современной Российской установки АЗТ с лазерным испарения и разработка методик работы на такой установке является актуальным.

\ifsynopsis
Этот абзац появляется только в~автореферате.
Для формирования блоков, которые будут обрабатываться только в~автореферате,
заведена проверка условия \verb!\!\verb!ifsynopsis!.
Значение условия задаётся в~основном файле документа (\verb!synopsis.tex! для
автореферата).
\else
\fi

% {\progress}
% Этот раздел должен быть отдельным структурным элементом по
% ГОСТ, но он, как правило, включается в описание актуальности
% темы. Нужен он отдельным структурынм элемементом или нет ---
% смотрите другие диссертации вашего совета, скорее всего не нужен.

{\aim} данной диссертационной работы является разработка комплекса атомно-зондовой томографии с фемтосекундным лазерным испарением с программными средствами автоматизации управления и контроля установки, а также развитие методик проведения атомно-зондовых исследований материалов, в том числе: методики определения условий проведения АЗТ исследований, методики сравнения результатов исследований на разных установках атомно-зондовой томографии.

Для~достижения поставленной цели необходимо было решить следующие {\tasks}:
\begin{enumerate}[beginpenalty=10000]
  \item Разработать установку атомно-зондовой томографии с фемтосекундным лазерным испарением;
  \item Разработать и ввести в эксплуатацию комплекс программ для контроля и управления установкой АЗТ;
  \item Оценить качество атомно-зондовых данных, полученных на разработанной установке в сравнении с другой установкой АЗТ;
  \item Разработать методику контроля условий испарения материала в процессе сбора данных по соотношению зарядностей алюминия в алюминиевых сплавах;
  \item Проверить возможность сравнения АЗТ данных, полученных на разных установках
  \item Разработать метод коррекции базового метода восстановления атомно-зондовых данных;
  \item Демонстрация возможностей разработанной установке по исследованию среднеуглеродистой стали, высокопрочной экономнолегированной стали и на алюминиевых сплавах.
\end{enumerate}


{\novelty}
Впервые в России спроектирована и создана установка атомно-зондовой томографии для изучения состава и структуры материалов с особенностями, характерный размер которых лежит в диапазоне от 1 до 500~нм. Разработаны методики контроля и воспроизводимости условий испарения. Предложена оригинальная методика коррекции 3D восстановления АЗТ данных. На созданной установке впервые в России проведены исследования алюминиевых сплавов, среднеуглеродистых, экономнолегированных сталей и других сплавов с целью получения уникальной информации о структуре и составе материалов.Показаны возможности установки по исследованию различных наномасштабных особенностей материалов, в том числе продемонстрирована возможность прецизионной характеризации таких объектов как: зоны Гинье-Перстона, радиационно-индуцированные кластеры, карбиды.

{\influence} Созданная установка АЗТ позволила провести ряд уникальных исследований структуры и состава материалов в рамках исследования радиационной стойкости материалов ядерной техники. Полученный опыт при разработке установки позволит модернизировать имеющиеся атомно-зондовые томографы и разрабатывать новые томографы в интересах научных институтов России, занимающихся материаловедением. Были проведены исследования для прогноза радиационной стойкости перспективных дисперсно-упрочненных оксидами (ДУО) сталей. Данные работы могут иметь практическое значение для научных организаций: ВНИИНМ им. Бочвара, ЦНИИКМ <<Прометей>>, НИТУ МИСиС, ГК Росатом. Разработанная установка применялась для получения уникальных результатов в работах по грантам РНФ (№23-79-10147, №20-79-10373, №22-29-01279, №17-19-01696), РФФИ (№18-38-00859, №18-32-20174).

\clearpage
{\defpositions}
\begin{enumerate}[beginpenalty=10000] % https://tex.stackexchange.com/a/476052/104425
  \item Первая в России установка атомно-зондовой томографии с фемтосекундным лазерным испарением отвечающий основным требования, предъявляемым методом атомно-зондовой томографии к конфигурации и работе оборудования, позволяет проводить атомно-зондовые исследования широкого спектра материалов  и получать информацию о трехмерном распределении элементов в объеме образца с разрешением, близким к атомному.
  \item Оригинальная методика коррекции атомно-зондовых данных по атомной плотности материала для компенсации ошибки восстановления трехмерных координат, обеспечивающая восстановление 3D координат атомов точнее, чем стандартные алгоритмы обработки.     
  \item Разработана методика контроля условий испарения для разных атомно-зондовых томографах с использованием метрики соотношения зарядностей одно- и двухзарядных пиков алюминия для алюминиевых сплавов, тем самым, повышая качество и воспроизводимость результатов исследований.
  \item Методика сопоставления и сравнения условий испарения на разных установках атомно-зондовой томографии с использованием метрики соотношения зарядностей основного химического элемента материала демонстрирует возможность проводить и сопоставлять АЗТ данные, полученные на разных установках АЗТ.
  \item Результаты исследования состава и структуры материалов с помощью атомно-зондовой томографии:
  \begin{itemize}
  	\item Получены атомные карты химических элементов для сплава Al-Mg-Si. Измерены плотность и характерные размеры включений Mg-Si. Получен состав нано-размерных включений
  	\item Исследованы сегрегации атомов углерода(кластеры) и крупные карбидные частицы в образцах среднеуглеродистой стали после различных температур отпуска.
  	\item Подтверждено наличие кластеров углерода в \\ экономнолегированной стали при высокотемпературном отпуске
  	\item Показана возможность анализа профилей концентраций на когерентной и полу-когерентной границах $\theta '$-фазы сплава алюминия Al-Si-Cu-Sn.
  \end{itemize}
  
  
\end{enumerate}

{\reliability} полученных результатов и выводов диссертационной работы обусловлена: применением общепринятых подходов к разработке и проектированию научных установок; использованием современных узлов и приборов в составе установки; соответствием качественно и количественно результатов, полученных с помощью АЗТ в сравнении с другими методиками. Полученные характеристики установки (пространственное разрешение и разрешение по массе) соответствуют результатам других разработчиков атомно-зондовых томографов.


{\probation}
Основные результаты работы докладывались на российских и международных конференциях:
\begin{itemize}
	\item Международная конференция <<APT$\&$M>> (Германия, Штутгарт, 2014);
	\item Международная конференция «European Atom Probe Workshop» (Швеция, Гётеборг, 2017; Германия, Дюссельдорф, 2018; Франция, Руан, 2019);
	\item Международный Уральский семинар <<Радиационная физика металлов и сплавов>> (Россия, Кыштым, 2015, 2017, 2019);
	\item Международная молодежная научная школа-конференция <<Современные проблемы физики и технологий>> (Москва, НИЯУ МИФИ 2016 год);
	\item Курчатовская молодежная школа (Россия, Москва, 2014, 2016, 2017);
	\item Всероссийская научно-практическая конференция <<Научное приборостроение – современное состояние и перспективы развития>> (Россия, Казань, 2018);	
	\item Международная конференция <<Деформация материалов и разрушение материалов и наноматериалов>> (Россия, Москва, 2019); 
	\item Научная конференция <<ИТЭФ – научные итоги года>> (Россия, Москва, 2019);
	\item Молодежная конференция по теоретической и экспериментальной физике (НИЦ КИ – ИТЭФ, Москва, Россия, 2016, 2017, 2018, 2019, 2020, 2021);
	\item Четвертый междисциплинарный научный форум с международным участием <<Новые материалы и перспективные технологии>> (Россия, Москва, 2018);
	\item Российская ежегодная конференция молодых научных сотрудников и аспирантов <<Физико-химия и технология неорганических материалов>> (Россия, Москва, 2018);
	\item Всероссийская научно-практическая конференция с международным участием <<Научное приборостроение – перспективы разработки, создания, развития и использования>> (Россия, Ростов-на-Дону, 2024);
	
\end{itemize}
Основное содержание работы и её результаты опубликованы в 9 печатных работах, из них 6 публикаций в изданиях, рекомендованных ВАК РФ, в том числе 5 статей, входящих в базы данных Scopus и Web of Science, и 3 тезисов в сборниках трудов российских и международных научных конференциях.

{\contribution} Диссертант Лукьянчук А.А. внес решающий вклад в создание, эксплуатацию и проведение экспериментов на первом в России атомно-зондовом томографе с лазерным испарением – ПАЗЛ-3D. Диссертант внес решающий вклад в разработку схему установки и компоновку узлов; систему загрузки/выгрузки образцов, систему охлаждения образца в анализационном объеме. Он участвовал в  пуско-наладочных работах, непосредственно проводил эксперименты по исследованию различных материалов и обрабатывал полученные данные. Диссертант также участвовал в разработке ПО обработки АЗТ данных (KVANTM-3D). ПО контроля и управления установкой ПАЗЛ-3D было разработано и использовано в исследованиях непосредственно диссертантом. В процессе работы на установке диссертантом была создана оригинальная методика коррекции восстановления 3D данных. Диссертант непосредственно участвовал в подготовке всех публикаций по теме квалификационной работы.










