
{\actuality} Разработка новых материалов для ядерной техники, микроэлектронике и других высокотехнологичных направления требует учета не только микроструктурных особенностей материала, но и информацию об атомном распределении химических элементов. Эти сведения составляют основу для описания радиационной стойкости, прочности и других макроскопических характеристик материалов. Кроме того, информация эволюции нано-размерных включений в материале применяется в интересах разработки и интерпретации новых эффектов на масштабе, близком к атомному.
Для контроля морфологии и химического состава наноразмерных объектов наилучшим образом подходит методика атомно-зондовой томографии. Практически все ведущие зарубежные ядерные центры оснащены такими приборами, включая аналитические центры крупных международных университетов (например, Окриджская национальная лаборатория (США), Оксфордский университет (Англия), Руанский университет (Франция), Институт технологий Карлсруэ (Германия), Университет Тохоку (Япония), Пекинский технологический университет (Китай) и др. \cite{APTlist}). Необходимо отметить, что атомно-зондовая томография – единственная методика, позволяющая получить трехмерное изображение химической структуры материала с атомарным разрешением. На сегодняшний день для аналогичных целей используются иные методики анализа, например, электронная микроскопия (просвечивающая, сканирующая), малоугловое рассеяние нейтронов, позитронная аннигилляционная спектроскопия, Мёссбауэровская спектроскопия. При определенных условиях указанные методики могут обеспечивать практически атомарное разрешение (просвечивающая электронная микроскопия высоко разрешения) или иметь высокое разрешение по массе (вторично-ионная масс-спектрометрия). Но ни одна из этих методик не может позволить получить истинно трехмерное отображение структуры материала с разрешением, близким к атомарному, одновременно с определением химической природы каждого зарегистрированного атома \cite{GaultBOOK}. С помощью атомно-зондовой томографии возможно исследовать широкий спектр материалов, например: структуру и трехмерное распределение атомов для материалов полупроводниковой промышленности \cite{Ulfig23} и биологические материалы, такие как коллагеновые структуры в кости \cite{Lee21}. Также, с помощью данной методики, можно проводить сложный корреляционный анализ структуры и состава материалов, например, с помощью просвечивающей электронной микроспорией(ПЭМ) и АЗТ исследуется один и тот же образец с карбидными включениями \cite{Liebscher18}.
В России, до недавних пор, инструментальная база для атомно-зондовых исследований была представлена всего двумя установками. Первая - в Институте Теоретической и Экспериментальной физики (сейчас является частью НИЦ "Курчатовский институт")(ссылка). Это довольно старая установка, разработанная французскими учеными в 2000-х годах и ограниченная по своим возможностям. Вторая - в НИЦ "Курчатовский институт", уже современный комплекс АЗТ исследований, но имеющий рад ограничений: ограниченный перечень материалов, возможных для исследований; проприетарное программное обеспечение, отсутствие лазерного испарения. Столь малое количество установок АЗТ видится недостаточным для ряда амбициозных целей отечественных научно-исследовательских институтов и предприятий. В связи с этим создание современной установки АЗТ с лазерным испарением и разработка методик работы на данной установке видятся актуальными.

\ifsynopsis
Этот абзац появляется только в~автореферате.
Для формирования блоков, которые будут обрабатываться только в~автореферате,
заведена проверка условия \verb!\!\verb!ifsynopsis!.
Значение условия задаётся в~основном файле документа (\verb!synopsis.tex! для
автореферата).
\else
\fi

% {\progress}
% Этот раздел должен быть отдельным структурным элементом по
% ГОСТ, но он, как правило, включается в описание актуальности
% темы. Нужен он отдельным структурынм элемементом или нет ---
% смотрите другие диссертации вашего совета, скорее всего не нужен.

{\aim} данной диссертационной работы является разработка и запуск комплекса атомно-зондовой томографии с фемтосекундным лазерным испарением, в том числе: , проведение исследований перспективных материалов, в том числе конструкционных сталей.%; разработка собственного ПО

Для~достижения поставленной цели необходимо было решить следующие {\tasks}:
\begin{enumerate}[beginpenalty=10000] % https://tex.stackexchange.com/a/476052/104425
  \item Разработать установку атомно-зондовой томографии с фемтосекундным лазерным испарением с характеристиками;
  \item Разработать методику проведения исследований сталей, в том числе получить базовые зависимости качества данных от основных параметров установки АЗТ;
  \item Оценить качество атомно-зондовых данных, полученных на разработанной установке в сравнении с другой установкой АЗТ;
  \item Разработать методику контроля условий испарения материала в процессе сбора данных;
  \item Разработать и внедрить метод коррекции базового метода восстановления атомно-зондовых данных;
  \item Апробация установки при исследовании уникальных материалов (Al-Mg-Si, сталь?, алюминий?).
\end{enumerate}


{\novelty}
Впервые в России спроектирована и создана установка атомно-зондовой томографии для изучения состава и структуры материалов с особенностями, характерный размер которых лежит в диапазоне от 1 до 500 нм. На созданной установке впервые в России проведены исследования алюминиевых сплавов с помощью атомно-зондовой томографии. Разработана методика проведения исследований сталей для созданной установки. Предложена оригинальная методика коррекции 3D восстановления АЗТ данных. Показаны возможности установки по исследованию различных материалов, в том числе продемонстрирована возможность прецизионной характеризуем таких объектов как: зоны Гинье-Перстона, радиационно-индуцированные кластеры, петли(?), карбиды(?).

{\influence} Созданная установка АЗТ позволила провести ряд уникальных исследований структуры и состава материалов в рамках исследования радиационной стойкости материалов ядерной техники. Полученный опыт при разработке установки позволит модернизировать имеющиеся атомно-зондовые томографы с полевым испарением и разрабатывать новые томографы в интересах научных институтов атомной отрасли России. Были проведены исследования для прогноза радиационной стойкости перспективных ДУО сталей. Данные работы могут иметь практическое значение для научных организаций: ВНИИНМ им. Бочвара, ЦНИИКМ "Прометей", НИТУ МИСиС. Разработанная установка применялась для получения уникальных результатов в работах по грантам РНФ (номера?), РФФИ, МИФИ СНИ.

\clearpage
{\defpositions}
\begin{enumerate}[beginpenalty=10000] % https://tex.stackexchange.com/a/476052/104425
  \item Первая в России установка атомно-зондовой томографии с фемтосекундным лазерным испарением с пространственным разрешением не хуже 4 Ангстрем и разрешением по массе не хуже 500 отн. ед.
  \item Оригинальная методика коррекции атомно-зондовых данных по атомной плотности материала для компенсации ошибки восстановления трехмерных координат.     
  \item Создана оригинальная методика сравнения качества данных на разных атомно-зондовых томографах.
  \item Методика контроля условий испарения в процессе сбора данных по соотношению зарядностей основного элемента исследуемого материала
  \item Результаты исследования структуры материалов с помощью атомно-зондовой томографии:
  \begin{itemize}
  	\item Получены атомные карты химических элементов для сплава Al-Mg-Si. Измерены плотность и характерные размеры включений Mg-Si. Получен состав нано-размерных включений
  	\item 
  	\item 
  \end{itemize}
  
  
\end{enumerate}

{\reliability} полученных результатов и выводов диссертационной работы обусловлена: применением общепринятых подходов к разработке и проектированию научных установок; использованием современных узлов и приборов в составе установки; соответствием качественно и количественно результатов, полученных с помощью АЗТ в сравнении с другими методиками. Результаты в части характеристик установки(пространственное разрешение и разрешение по массе) находятся в соответствии с результатами, полученными другими авторами.


{\probation}
Основные результаты работы докладывались на российских и международных конференциях:
\begin{itemize}
	\item Международная конференция «APT$\&$M» (Германия, Штутгарт, 2014);
	\item Международная конференция «7th European Atom Probe Workshop» (Швеция, Гётеборг, 2017);
	\item 10-ы и 11-ый Международного Уральского семинара «Радиационная физика металлов и сплавов» (Россия, Кыштым, 2015, 2017, 2019);
	\item Международная молодежная научная школа-конференция «Современные проблемы физики и технологий» (Москва, НИЯУ МИФИ 2016 год);
	\item 12-ая, 13-ая, 14-ая Курчатовская молодежная школа (Россия, Москва, 2014, 2016, 2017);
	\item 2-я Всероссийская научно-практическая конференция «Научное приборостроение – современное состояние и перспективы развития» (Россия, Казань, 2018);
	\item Международная конференция 10th European Atom Probe Workshop (Германия, Дюссельдорф, 2018);
	\item Международная конференция European Atom Probe Tomography Workshop (Франция, Руан, 2019);
	\item 8-ая Международная конференция «Деформация материалов и разрушение материалов и наноматериалов» (Россия, Москва, 2019); 
	\item Научная конференция «ИТЭФ – научные итоги года» (Россия, Москва, 2019);
	\item Молодежная конференция по теоретической и экспериментальной физике (НИЦ КИ – ИТЭФ, Москва, Россия, 2016, 2017, 2018, 2019, 2020, 2021);
	\item Четвертый междисциплинарный научный форум с международным участием "Новые материалы и перспективные технологии" (Россия, Москва, 2018);
	\item ХV Российская ежегодная конференция молодых научных сотрудников и аспирантов "Физико-химия и технология неорганических материалов" (Россия, Москва, 2018);
	
\end{itemize}

{\contribution} Диссертант Лукьянчук А.А. внес решающий вклад в создание, эксплуатацию и проведение экспериментов на первом в России атомно-зондовом томографе с лазерным испарением – ПАЗЛ-3D. Диссертант лично разрабатывал схему установки и компоновку узлов; систему загрузки/~выгрузки образцов, систему охлаждения образца в анализационном объеме. Он непосредственно проводил пуско-наладочные работы, проводил эксперименты по исследованию различных материалов и обрабатывал полученные данные. Диссертант также участвовал в разработке ПО обработки АЗТ данных (KVANTM-3D). В процессе работы на установке диссертантом создана оригинальная методика коррекции восстановления 3Д данных и зарегистрировано НОУ-ХАУ по улучшению качества данных, получаемых на установке ПАЗЛ-3D. Диссертант непосредственно участвовал в подготовке всех публикаций по теме квалификационной работы.






